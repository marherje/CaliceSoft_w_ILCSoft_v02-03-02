\hypertarget{root_lib_Idea}{}\section{The idea}\label{root_lib_Idea}
Root's {\ttfamily T\-Tree\-::\-Draw()} command is very powerful and flexibe and allows the user to create many plots in a short time. The lib\-Dead\-And\-Noisy\-Tools library can be loaded in C\-I\-N\-T and provides a class to create the tree with the rms and mean values of the channel spectra (see \hyperlink{index_Introduction}{Introduction}). In addition there are convenience functions to create the default plots and to compare two runs.

Like this it is possible to check with just a few lines at the C\-I\-N\-T prompt whether a run is O\-K and which channels are bad, while having full access to all the underlying information and interactively defining all cuts.

Currently the library provides five classes\-: \begin{DoxyItemize}
\item \hyperlink{class_simple_mapper}{Simple\-Mapper} A helper class which contains a map (slot,frontent)-\/$>$(module,layer) \item \hyperlink{class_t_channel_spectrum_properties}{T\-Channel\-Spectrum\-Properties} A data class to put into a T\-Tree. \item \hyperlink{class_spectrum_properties_run_info}{Spectrum\-Properties\-Run\-Info} The main class for information about one run \item \hyperlink{class_run_comparator}{Run\-Comparator} A class to facilitate the comparison of two runs \item \hyperlink{class_channel_expert_quality}{Channel\-Expert\-Quality} A class defining the status quality words used in this code\end{DoxyItemize}
The idea is the following\-: The \hyperlink{class_spectrum_properties_run_info}{Spectrum\-Properties\-Run\-Info} will provide a tree with the spectrum properties. The T\-Tree\-::\-Draw() command is very powerful and flexibe and allows the user to create many plots in a short time. The tree is filled when instantiating the \hyperlink{class_spectrum_properties_run_info}{Spectrum\-Properties\-Run\-Info} object. The constructor just gets the run number and the file name of the A\-H\-C.\-cfg file. It opens a file named {\itshape run\-Number}.root to read the histograms.\hypertarget{root_lib_Usage}{}\section{Usage}\label{root_lib_Usage}
\hypertarget{root_lib_Cplusplus}{}\subsection{in normal C++ programms}\label{root_lib_Cplusplus}
Just include the header files you need and link the lib\-Dead\-And\-Noisy\-Tools.\-so\hypertarget{root_lib_Root}{}\subsection{in R\-O\-O\-T's C\-I\-N\-T}\label{root_lib_Root}
Load the lib\-Dead\-And\-Noisy\-Tools.\-so with \begin{DoxyVerb}.L libDeadAndNoisyTools.so \end{DoxyVerb}
 or put \begin{DoxyVerb}gSystem->Load("libDeadAndNoisyTools.so"); \end{DoxyVerb}
 into your rootlogon.\-C. After this the classes are known to C\-I\-N\-T and you can use them interactively.\hypertarget{root_lib_Example}{}\subsection{Example}\label{root_lib_Example}
\begin{DoxyItemize}
\item \mbox{[}1\mbox{]} Load the shared library \item \mbox{[}2\mbox{]} Instantiate the \hyperlink{class_spectrum_properties_run_info}{Spectrum\-Properties\-Run\-Info} object. The input file name is {\ttfamily run\-Number.\-root} if no other name is given (360759.\-root in this case). \item \mbox{[}3\mbox{]} Get a pointer to the tree. \item \mbox{[}4\mbox{]} For convenience\-: Change the marker stype of the tree to {\ttfamily +}.\end{DoxyItemize}
Afterwards you can draw the contents of the tree, e. g. \begin{DoxyItemize}
\item \mbox{[}5\mbox{]} Histogram the mean values for all channels in slot 19 \item \mbox{[}6\mbox{]} Draw a graph rmv vs. channel for all channels in module 9\end{DoxyItemize}
\begin{DoxyVerb}root [1] .L libDeadAndNoisyTools.so
root [2] SpectrumPropertiesRunInfo spri(360759,"AHCforCERN2010.cfg");
root [3] TTree *t = spri.getSpectrumPropertiesTree()
root [4] t->SetMarkerStyle(2);
root [5] t->Draw("_mean","_slot==19")
root [6] t->Draw("_rms:_channel+_chip*18","_module==9")
\end{DoxyVerb}


For more examples using the \hyperlink{class_spectrum_properties_run_info}{Spectrum\-Properties\-Run\-Info} have a look at the \hyperlink{_examples}{Examples} section. A complete list of all function can be found in the \hyperlink{class_spectrum_properties_run_info}{Spectrum\-Properties\-Run\-Info} class documentation. 
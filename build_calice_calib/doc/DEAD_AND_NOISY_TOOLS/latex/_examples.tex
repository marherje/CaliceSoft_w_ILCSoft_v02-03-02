These are some examples for the interactive use at R\-O\-O\-T's C\-I\-N\-T prompt.

All examples assume that you have the lib\-Dead\-And\-Noisy\-Tools library loaded. 
\begin{DoxyCode}
~> export LD\_LIBRARY\_PATH=$LD\_LIBRARY\_PATH:$\{PATH\_TO\_CALICE\_CALIB\}/lib
~> root -l
root [0] gSystem->Load(\textcolor{stringliteral}{"libDeadAndNoisyTools.so"})
\end{DoxyCode}


Note\-: The {\ttfamily export} command is for bash, you might have to adapt it if you are using a different shell.

If you need it regularly you might want to put the {\ttfamily g\-System-\/$>$Load()} command into your rootlogon.\-C\hypertarget{_examples_usingSpectrumPropertiesRunInfo}{}\section{Having a closer look at one run using Spectrum\-Properties\-Run\-Info}\label{_examples_usingSpectrumPropertiesRunInfo}
The input for the \hyperlink{class_spectrum_properties_run_info}{Spectrum\-Properties\-Run\-Info} class is a root file with the spectra for each channel. Usually this is in a files called {\ttfamily runnumber.\-root}. Currently also a file with the A\-H\-C config is needed for the frontend/slot $<$-\/$>$ module mapping. This is subject to be removed once the mapping in ahc\-Bin\-Hst ist fully working.\hypertarget{_examples_plotRun}{}\subsection{Plotting the rms of all channels in a run}\label{_examples_plotRun}
For this we use the T\-Tree\-:.Draw() command from root. The \hyperlink{class_spectrum_properties_run_info}{Spectrum\-Properties\-Run\-Info} class provides the tree for you. All you need is the input file {\ttfamily R\-U\-N\-N\-U\-M\-B\-E\-R.\-root}. You simply create the \hyperlink{class_spectrum_properties_run_info}{Spectrum\-Properties\-Run\-Info} object \mbox{[}1\mbox{]} and get the tree from it \mbox{[}2\mbox{]}. For drawing T\-Graphs (2\-D plots) it is convenient to change the marker style from a one pixel dot to something larger \mbox{[}3\mbox{]}. 
\begin{DoxyCode}
root [1] \hyperlink{class_spectrum_properties_run_info}{SpectrumPropertiesRunInfo} spri(360759,\textcolor{stringliteral}{"AHCforCERN2010.cfg"});
root [2] TTree *t = spri.getSpectrumPropertiesTree()
root [3] t->SetMarkerStyle(2);
\end{DoxyCode}


Now you can draw any information in the Tree with just one line, incl. cuts. \begin{DoxyItemize}
\item \mbox{[}4\mbox{]} Histogram the mean values for all channels \item \mbox{[}5\mbox{]} Draw a graph rmv vs. channel for all channels in module 9 
\begin{DoxyCode}
root [4] t->Draw(\textcolor{stringliteral}{"\_mean"},\textcolor{stringliteral}{"\_slot==19"})
root [5] t->Draw(\textcolor{stringliteral}{"\_rms:\_channel+\_chip*spri.getNChannelsPerChip()"},\textcolor{stringliteral}{"\_module==9"}) 
\end{DoxyCode}
 Note the usage of \hyperlink{class_spectrum_properties_run_info_a3d9d8e0ae2cef40794561409120e257c}{Spectrum\-Properties\-Run\-Info\-::get\-N\-Channels\-Per\-Chip()}. There is also the Spectrum\-Properties\-Run\-Info\-::get\-N\-Chips\-Per\-Modue() function. You can use this to calculate the global channel number $( moduleID \cdot nChipsPerModule + chipID ) \cdot nChannelsPerChip + channelNumber $.\end{DoxyItemize}
\hypertarget{_examples_plotDead}{}\subsection{Plotting the rms of only the dead channels}\label{_examples_plotDead}
\hypertarget{_examples_printDead}{}\subsection{Printing a list of dead channels in one run}\label{_examples_printDead}
